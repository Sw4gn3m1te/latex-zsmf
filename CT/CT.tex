\documentclass[12pt,a4paper]{article}
\usepackage[utf8]{inputenc}
\usepackage[german]{babel}
\usepackage[T1]{fontenc}
\usepackage{times}
\usepackage{graphicx}
\usepackage{url}
\usepackage{color}
\usepackage{setspace}
\usepackage{enumerate}
\usepackage{amsmath}
\usepackage{amsfonts}
\usepackage{amssymb}
\usepackage{float}
\usepackage{stix}
\title{Complexity Theory}
\author{Henrik Tscherny}
\begin{document}
\maketitle
\tableofcontents

\section{Definitionen}
\subsection{Entscheidbarkeit und Erkennbarkeit}
\begin{itemize}
\item Sei M eine TM mit dem Eingabealphabet $\Sigma$
\item Sei $L(M) := \{w \in \Sigma^* \vert M akzeptiert w\}$
\item Ein Sprache L ist \textbf{erkennbar} gdw. es eine TM gibt welche diese Sprache erkennt ($L = L(M)$)
\item Außerdem ist eine Sprache \textbf{erkennbar} gdw. es einen Enumerator E mit $G(E) = L$
\item Ist eine Sprache \textbf{erkennbar}, dann existiert ein Enumerator E für L welcher jedes Wort in L nur genau einmal ausgibt
\item Eine Sprache L ist \textbf{entscheidbar} gdw. sie erkennbar ist und die TM auf jedem Input hält
\item Eine Sprache L ist \textbf{erkennbar} gdw. es einen Enumerator E gibt welcher die Wörter in aufsteigender Länge aufzählt
\item Eine Sprache L ist \textbf{co-semi-entscheidbar} gdw. $\bar{L}$ semi-entscheidbar ist
\item Ist L semi-entscheidbar und co-semi-entscheidbar, so ist L entscheidbar\\
Beweis:
\begin{itemize}
\item Sei $TM_L$ eine TM welche L erkennt
\item Sei $TM_{\bar{L}}$ eine TM welche $\bar{L}$ erkennt
\item Simuliere $TM_L$ und $TM_{\bar{L}}$
\item eine von beiden muss halten, da L semi-entscheidbar und co-semi-entscheidbar ist
\item somit kann $w \in L$ entschieden werden
\end{itemize}
\item Es gibt Probleme Welche weder semi- npch co-semi-entscheidbar sind z.B. TM Äquivalenz (Beweis siehe Many-One-Reduktionen)
\end{itemize}
Note: Erkennbar = semi-entscheidbar

\subsection{Enumerator (Aufzähler)}
Eine Multi-Band TM M mit:
\begin{itemize}
\item M hat ein \textbf{write-only output Band}, auf dem der Head nur nach Links laufen kann
\item M hat ein $\#$ Symbol welches die Wörter auf dem Output Band trennt
\item Die von \textbf{M erzeugte Sprache} sei die Sprache aller Wörter welche irgendwann auf dem Output Band zwischen Zwei $\#$ auftauchen
\item M \textbf{startet} auf einem \textbf{leeren Band}
\end{itemize}

\section{Arten von TM's}
\subsection{einseitig begrenzte einband TM}
\paragraph{Definition}
\flushleft
\begin{itemize}
\item $M=(Q, \Sigma, \Gamma, \delta, q_0, q_{accept}, q_{reject})$
\item Q: Zustandsmenge
\item $\Sigma$: Eingabealphabet (ohne \textvisiblespace)
\item: $\Gamma$: Bandalphabet mit $\Sigma \cup \{$\textvisiblespace$\} \subseteq \Gamma$
\item $\delta$: Übergangsfunktion mit $\delta: Q \times \Gamma \rightarrow Q \times \Gamma \times \{L, R\}$
\item $q_0$: Startzustand
\item $q_{accept}$: akzeptierender Endzustand mit $q_{accept} \in Q$
\item $q_{reject}$: ablehnender Endzustand mit $q_{reject} \in Q, q_{accept} \neq q_{reject}$
\end{itemize}

\subsection{k-Band TM}
\paragraph{Definition}
\flushleft
\begin{itemize}
\item $M=(Q, \Sigma, \Gamma, \delta, q_0, q_{accept}, q_{reject})$
\item $\delta: Q \times \Gamma^k \rightarrow Q \times \Gamma^k \times \{L, R, N\}^k$
\end{itemize}
\paragraph{Äquivalenzbeweis}
\flushleft
\begin{itemize}
\item Sei M eine k-Band TM und S eine TM welche M simuliert
\item Schreibe den Inhalt der k Bänder von M hintereinander jeweils getrennt durch ein Trennzeichen ($\#$)
\item Die Head-Positionen von M werden durch spezielle Zeichen in S dargestellt ($\dot{x}$)
\end{itemize}

\subsection{Oracle TM}
\paragraph{Definition}
\flushleft
\begin{itemize}
\item besitzt ein spezielles \textbf{Oracle Band}
\item hat spezielle Zustände $q_?, q_{yes}, q_{no}$
\item Wenn die OTM $q_?$ erreicht, springt diese zu $q_{yes}$ oder $q_{no}$, je nachdem ob der Inhalt des Oracle Bandes in O ist
\item O kann ein beliebig schweres Problem sein, sogar unentscheidbar
\item Das Oracle zu benutzen benötigt lediglich einen Schritt
\item Ein Oracle zu Komplementieren hat keine Auswirkung
\end{itemize}

\section{Reduktionen}
\subsection{Turing-Reduktionen}
\paragraph{Definition}
\flushleft
ein Problem P ist \textbf{Turing reduzierbar} auf ein Problem Q (P $\leq_T$ Q), wenn P durch eine OTM $M^Q$ mit Oracle Q entschieden werden kann\\
Turing-Reduktion kann genutzt werden, um Unentscheidbarkeit zu beweisen\\
\textbf{Ist P unentscheidbar und P $\leq_T$ Q dann ist Q unentscheidbar}\\
Beweis mittels Kontrapositiv:
\begin{itemize}
\item Nehme an P $\leq_T$ Q mit Q ist entscheidbar
\item Man kann die benötigte OTM als normale TM konstruieren, so dass gilt P $\leq_T$ Q
\item daraus folgt das P entscheidbar ist
\end{itemize}
Note: Ist Q semi-entscheidbar und P $\leq_T$ Q, dann kann oder kann P nicht semi-entscheidbar sein


\paragraph{Beispiel-Reduktion}
\flushleft
Reduzieren des $\epsilon$-Halteproblems auf das Halteproblem\\
(Hält eine TM M auf einem leeren Inputband ?)
\begin{itemize}
\item Definiere eine OTM wie folgt:
\begin{itemize}
\item Input: Eine TM M und ein Wort w
\item Konstruiere eine TM $M_w$ wie folgt:
\begin{enumerate}
\item Lösche alles vom Inputtape und schreibe w darauf
\item Bearbeite den Input wie M
\end{enumerate}
\item Löse das $\epsilon$-Halteproblem für $M_w$ mit einem Oracle
\item Output: Ergebnis des $\epsilon$-Halteproblem
\end{itemize}
\item $\Rightarrow$ \textbf{Unentscheidbar}
\end{itemize}

\subsection{Many-one-Reduktion (mapping Reduction)}
\paragraph{Definition}
\flushleft
Eine Sprach P ist \textbf{many-one-reduzierbar} auf eine Sprache Q (P $\leq_m$ Q), wenn es eine \textbf{totale berechenbare Funktion} f gibt, so dass $f: \Sigma^* \rightarrow \Sigma^*$, so dass\\
$\forall w\in \Sigma^*: w\in P \, \Leftrightarrow \, f(w) \in Q$\\
\begin{itemize}
\item Many-one-Reduktion \textbf{erhält (co-)semi-entscheidbarkeit} (anders als die Turing-Reduktion)
\item P $\leq_m$ Q $\Rightarrow$ P $\leq_T$ Q\\
Beweis:
\begin{itemize}
\item Man erhält eine OTM mit Oracle Q, welches P erkennt wie folgt:
\begin{itemize}
\item Input w $\rightarrow$ berechne f(w)
\item Frage das Oracle und gebe das Ergebnis zurück (ja = akzeptieren, nein = verwerfen)
\end{itemize}
\end{itemize}
\item gilt P $\leq_m$ Q und Q ist entscheidbar, dann ist P entscheidbar
\item gilt P $\leq_m$ Q und Q ist semi-entscheidbar, dann ist P semi-entscheidbar\\
Beweis:
\begin{itemize}
\item gegeben eine TM welche Q erkennt, man erhält eine TM welche P erkennt wie folgt:
\begin{itemize}
\item Auf dem Input w, berechne f(w)
\item Simuliere die TM für Q und gebe das Ergebnis zurück
\end{itemize}
\end{itemize}
\end{itemize}

\paragraph{Beispiel-Reduktion}
Zwei TM's M und N sind äquivalent wenn sie die gleiche Sprache erkennen (L(M) = L(N))\\
TM Äquivalenz ist unentscheidbar:\\
Beweis:
\begin{itemize}
\item definiere f, so dass $w\in \epsilon-\text{Halting}$ gdw. $f(w) \in \text{Äquivalenz}$
\item Sei $M_a$ eine TM welche alle Inputs akzeptiert (hält auf jedem beliebigen w)
\item Definiere für eine TM M eine TM $M^*$ wie folgt:
\begin{itemize}
\item Simuliere M auf $\epsilon$
\item wenn M hält akzeptiere
\end{itemize}
\item $M^*$ akzeptiert also alle Inputs sofern, M auf $\epsilon$ hält
\item $M^*$ ist äquivalent zu $M_a$ gdw. M auf $\epsilon$ hält
\item $\displaystyle f(w) = \begin{cases} \langle M^*, M_a \rangle, & w= \langle M \rangle \text{\hspace{25pt}(valide Kodierung)} \\ \epsilon, & \text{ungültige Kodierung} \end{cases}$\\
$\rightarrow$ $M^*$ auf $M_a$ akzeptiert immer, da $M_a$ für für jeden beliebigen Input hält, somit auch auf $\epsilon$
\end{itemize}

Äquivalenz von TMs ist weder semi- noch co-semi-entscheidbar\\
Beweis:
\begin{itemize}
\item Wie gezeigt gilt $\epsilon$-Halteproblem $\leq_m$ Äquivalenz
\item Da das $\epsilon$-Halteproblem nicht co-semi-entscheidbar ist, ist Äquivalenz es auch nicht
\item Jedoch kann man zeigen, dass nicht-$\epsilon$-Halteproblem $\leq_m$ Äquivalenz
\begin{itemize}
\item Sei $M_\emptyset$ eine TM welche alle Inputs ablehnt (Gegenteil zu $M_a$)
\item Äquivalenz zu $M_a$ korrespondiert zu $\epsilon$-halten
\item Äquivalenz zu $M_\emptyset$ korrespondiert zu $\epsilon$-nicht-halten
\item $\displaystyle f(w) = \begin{cases} \langle M^*, M_\emptyset \rangle, & w= \langle M \rangle \text{\hspace{25pt}(valide Kodierung)} \\ \langle M^*, M_\emptyset \rangle, & \text{ungültige Kodierung} \end{cases}$
\end{itemize}
\end{itemize}


\section{Satz von Rice}
\paragraph{Trivialität}
\flushleft
\begin{itemize}
\item Sei P eine Menge von Sprachen
\item Eine Sprache L hat die Eigenschaft P wenn $L \in P$
\item P ist \textbf{nicht-trivial}, wenn es erkennbare Sprachen gibt welche P haben und solche die P nicht haben\\
$\rightarrow$ es muss Entscheidungsmöglichkeiten geben
\end{itemize}

\paragraph{Satz}
\flushleft
Ist P eine \textbf{nicht-triviale} Eigenschaft einer erkennbaren Sprache, dann ist folgendes Problem unentscheidbar:\\
$P-ness = \{\langle M \rangle \vert L(M) \in P\}$\\
d.h. Die Frage ob eine TM eine nicht-triviale Eigenschaft, hat ist unentscheidbar

\paragraph{Beweis}
\flushleft
Reduzieren auf $\epsilon$-Halteproblem
\begin{itemize}
\item Sei $\emptyset \not\in P$
\item Sei $M_L$ eine TM welche die Sprache L erkennt und L die Eigenschaft P hat (L $\in$ P)
\item Konstruiere für eine beliebige TM M eine TM $M^*$ wie folgt für einen Input $w\in \Sigma^*$:
\begin{enumerate}

\item Simuliere M mit leerem Input ($\epsilon$)
\item Hält M, dann simuliere $M_L$ auf w
\end{enumerate}
\item Dadurch gilt $L(M^*) = L \in P$ wenn M auf $\epsilon$ hält, da (2) lediglich $M_L$ auf w simuliert
\item Sonnst gilt $L(M^*) = \emptyset \not\in P$, da $M^*$ in (1) festhängt
\item \textbf{Die Überprüfung ob $\langle M^* \rangle \in P$ würde das $\epsilon$-Halteproblem entscheiden\\
$\Rightarrow$ Unentscheidbar}
\end{itemize}

\paragraph{Anwendungen}
\flushleft
Unter anderem kann für folgende Eigenschaften für eine beliebige TM M mit Rice gezeigt werden, das die Sprachen unentscheidbar sind:
\begin{itemize}
\item Leerheit
\item Endlichkeit
\item Entscheidbarkeit
\item Regularität
\item Kontextfreiheit
\item Wortproblem
\end{itemize}

\section{Rekursion}
Eine \textbf{Quine} ist ein Programm, welches wenn gestartet, ohne zusätzlichen Input seinen eigenen Sourcecode ausgibt und dann hält\\
Ziel: Konstruieren einer TM SELF welche sich selbst als Encoding zurück gibt
\begin{itemize}
\item Es gibt eine berechenbare Funktion $q: \Sigma^* \rightarrow \Sigma^*$, so dass $\forall w \in \Sigma^* : q(w) = print (\langle M \rangle)$\\
Der Teil mit 'für jedes Wort' heißt einfach: ignoriere den Input (Input-unabhängig)
\item Man kann eine TM $P_w$ konstruieren, welche für jedes Wort den Tapeinhalt mit w ersetzt
\item q kann man jetzt berechnen, indem man eine TM nimmt, welche mit w als input $P_w$ erzeugt und anschließend $\langle P_w \rangle$ ausgibt
\end{itemize}
Eine Quine kann in zwei Teile unterteilt werden:
\begin{itemize}
\item A: berechne den Sourcecode $\langle B \rangle$ eines Programms B
\item B: nutze $\langle B \rangle$ um folgendes zu printen:\\
Sourcecode $\langle A \rangle$ welcher $\langle B \rangle$ berechnet und den Sourcecode $\langle B \rangle$ selbst
\end{itemize}
A kann mit der zuvor konstruierten TM $P_{\langle B \rangle}$ implementiert werden\\
B funktioniert wie folgt auf einem Input $\langle M \rangle$:
\begin{itemize}
\item berechne $q(\langle M \rangle)$
\item konkatoniere die TM's gegeben durch $q(\langle M \rangle)$ und $\langle M \rangle$
\item gebe die Kodierung dieser Konkatenation zurück
\end{itemize}
die TM SELF istnun eine TM konstruiert durch B auf dem Input $\langle B \rangle$

\section{Zeitkomplexität}
\paragraph{Definition}
\flushleft
Sei M eine TM und f eine Funktion mit $f: \mathbb{N} \rightarrow \mathbb{R}^+$\\
M ist \textbf{f-time beschränkt}, wenn M auf jedem input $w \in \Sigma^*$ nach maximal $f(\vert w \vert)$ Schritten hält

\subsection{Landau-Symbole}
\paragraph{Big-O}
\begin{itemize}
\item Klassifiziert eine Funktion mit einer asymptotischen oberen Schranke\\
\item $f(n) = O(g(n)) \Leftrightarrow \exists c>0 \exists n_0 \in \mathbb{N} \forall n > n_0 : f(n) \leq c \cdot g(n)$
\item es gibt also ein Punkt ($n_0$) ab dem die Funktion f dauerhaft kleiner gleich der Funktion g ist, selbst wenn g mit einer beliebigen Konstante ($c$) multipliziert wird
\end{itemize}
Note: für \textbf{small-o} sagt man auch f wird asymtotisch von g dominiert\\

\textbf{Symboltabelle}:\\
\begin{tabular}{|c|c|c|}
\hline
Notation & $ C = \lim_{n\rightarrow \infty} \frac{f(n)}{g(n)}$ & Bedeutung \\
\hline
$f \in O(g)$ & $c < 0$ & $f\leq g$\\
\hline
$f \in \Omega(g)$ & $c >0$ & $f\geq g$\\
\hline
$f \in \Theta(g)$ & $0<c<\infty$ & $f=g$\\
\hline
$f \in o(g)$ & $c=0$ & $f<g$\\
\hline
$f \in \omega(g)$ & $c=\infty$ & $f>g$\\
\hline
\end{tabular}

\end{document}