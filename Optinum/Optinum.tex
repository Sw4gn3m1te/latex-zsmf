\documentclass[12pt,a4paper, hyperref]{article}
\usepackage[utf8]{inputenc}
\usepackage[german]{babel}
\usepackage[T1]{fontenc}
\usepackage{times}
\usepackage{graphicx}
\usepackage{url}
\usepackage{color}
\usepackage{setspace}
\usepackage{enumerate}
\usepackage{amsmath}
\usepackage{amsfonts}
\usepackage{amssymb}


\newcommand{\tbar}[1]{\ensuremath{\bar{#1}}}
\newcommand{\tsnake}[1]{\ensuremath{\tilde{#1}}}

\title{Zusammenfassung Optinum}
\author{Henrik Tscherny}

\begin{document}
\maketitle
\tableofcontents

\section{Basics}

\subsection{Begriffe}
\begin{itemize}
\item Zielfunktion f
\item zulässiger Bereich G
\item $x \in G$ zulässiger Punkt
\end{itemize}

\subsection{Wann ist ein(e) Punkt/Lösung optimal ?}

x* sei eine optimale Lösung in zulässigen Bereich gdw. 
\begin{itemize}
\item Es gibt keinen Wert f(x) welcher echt kleiner ist als f(x*)
\item x* liegt in G
\end{itemize}


 $f(x*) \leq f(x)$ für $x \in G$

\subsection{Wann ist ein Optimierungsproblem linear ?}
ein OP sei linear gdw.
\begin{itemize}
\item G ist durch lineare Bedingungen beschreibbar
\item die ZF kann durch $c^\intercal x$ beschrieben werden. e.g.
$f=\begin{bmatrix} 2 \\ -3 \\ 7\end{bmatrix}^\intercal x = \begin{bmatrix}
2x && -3x && 7x\end{bmatrix}$
\end{itemize}


\subsection{Was ist eine Relaxation ?}
eine Funktion g heißt Relaxation von f gdw.
\begin{itemize}
\item g(x) ist für alle x kleiner oder gleich f(x)
\item der optimale ZF der Relaxation ist der Schrankenwert
\item $g(x) \leq f(x)$
\end{itemize}


\subsection{Wann ist eine Lösung einer Relaxation auch Lösung der nicht-Relaxation ?}
Wenn gilt:
\begin{itemize}
\item  \tbar x löst die Relaxation
\item \tbar x liegt auch im zulässigen Bereich von f
\item f(\tbar x) = g(\tbar x)
\end{itemize}
Note: Verschiedene Relaxationen liefern verschieden gute Näherungswerte\\
Eine Relaxation Q1 mit größerem Schrankenwert als Q2 heißt 'stärker'


\subsection{Wann existiert eine globale optimale Lösung ?}
Es gibt eine globale Lösung des OP gdw.
\begin{itemize}
\item G ist kompakt (beschränkt und abgeschlossen)
\item G ist nicht leer
\item die ZF ist stetig 
\end{itemize}
How To:
\begin{itemize}
\item Stelle ZF nach einer beliebigen Variable um
\item Kann der ZF nun beliebig klein gewählt werden und trotzdem eine zugehörige Koordinate errechnet werden so gibt es kein Minimum
\end{itemize}


\subsection{Wann ist eine lokale Lösung zugleich eine globale Lösung ?}
Eine lokales Optimum ist auch das globale Optimum gdw.
\begin{itemize}
\item G ist eine konvexe Menge
\item f ist eine konvexe Funktion
\end{itemize}
Note: Ist f streng konvex so existiert höchstens genau eine Lösung,


\subsection{Was ist ein Kegelmenge}
Ein Kegelmenge sei:
\begin{itemize}
\item alle nicht negativen vielfachen der Elemente einer Menge
\item $x \in K \Rightarrow \forall\lambda\geq 0: \lambda x \in K$
\end{itemize}
des weiteren ist ein Kegel konvex wenn:
\begin{itemize}
\item die Summe zweier Punkte aus der Menge ebenfalls in der Menge liegen
\item $x, y \in K \Rightarrow x+y \in K$
\end{itemize}
ein Kegel der zulässigen Richtung eines Punktes \tsnake x ist:
\begin{itemize}
\item alle Punkte welche vom Punkt \tsnake x aus mit der Schrittweite t in die Richtung d, wobei der gesamte Weg dabei in G liegen muss
\item die Schrittweite ist abh. von \tsnake x selbst und der aus $d \in \mathbb{R}^n$ gewählten Richtung
\item $Z(\tilde{x}) = \{d \in \mathbb{R}^n | \exists \tilde{t} = \tsnake t (\tilde{t} x ,d) >0, sodass \hspace{5pt} \tilde{x} + td \in G \forall t \in [0, \tilde{t} ]\}$
\end{itemize}

\subsection{notwendige und hinreichende Optimalitätsbedingung}
notwendige Bedingung:
\begin{itemize}
\item der Gradient ist am Minimum \tsnake x für alle Richtungen > 0 $\rightarrow$ es gibt keine Richtung in welche man gehen kann, sodass der ZF-Wert kleiner wird
\item $\nabla f(\tilde{x})^\intercal (x-\tilde{x}) \geq 0 \forall x \in G$
\item ist $G \in \mathbb{R}^n$ so gilt das \tsnake x ein Minimum ist wenn der Gradient 0 ist
\item $\nabla f(\tilde{x}) = 0$
\end{itemize}

\flushleft hinreichende Bedingung:
\begin{itemize}
\item f ist konvex
\item f ist stetig differenzierbar
\item es existiert ein $\tilde{x} \in G$ welches die notwendigen Bedingungen erfüllt
\end{itemize}

\flushleft Sind die notwendigen und hinreichenden OB erfüllt, so ist \tsnake x ein globales Minimum

\subsection{Was ist eine polyedrische Menge ?}
Eine Menge heißt polyedrisch gdw.
\begin{itemize}
\item kann durch eine Menge linearer Restringtionen definiert werden.
\item die linearen Resringtionen bilden einen Polyeder
\item $G = \{ x \in G \mathbb{R}^n | Ax \leq b \}$ 
\end{itemize}
Note: eine polyedrische Menge ist konvex und abgeschlossen aber nicht immer beschränkt


\subsection{Was besagt das Lemma von Farkas ?}
das Lemma von Farkas besagt:
\begin{itemize}
\item Es ist eine m x n Matrix A gegeben
\item dann ist genau ein der folgenden Gleichungen Lösbar:
\begin{itemize}
\item $Az \leq 0, a^\intercal z >0$
\item $A^\intercal u = a, u \geq 0$
\end{itemize}
\end{itemize}

\section{Ecken und Simplex}
\subsection{Definitionen}
\begin{itemize}
\item die allgemeine Form einer OP ist: $z = c^\intercal x \rightarrow min, x \in G .= \{x \in \mathbb{R}^n : Ax=b,x\geq 0\}$
\item G ist dabei eine polyedrische Menge
\item Alle Spalten der Basismenge sind linear unabhängig
\item $I_B$ ist die Menge der Basisindices
\item $_N$ ist die Menge der Nichtbasisindices mit $I_N = I\textbackslash I_N$
\item $A_B = (A^{i}_{i\in I_B}$, $A_N = (A^{i}_{i\in I_N}$
\item $c_B = (c_{i})_{i\in I_B}$, $c_N = (c_{i})_{i\in I_N}$
\item $x_B = (x_{i})_{i\in I_B}$, $x_N = (x_{i})_{i\in I_N}$
\end{itemize}
\subsection{Basislösung und Ecken}
\textbf{Basislösungen}:\\
\begin{itemize}
\item Ein Punkt x heißt Basislösung zu wenn eine gültige Belegung für jedes $x_B$ gefunden werden kann und dabei jedes $x_N=0$ ist\\
\item Eine Basislösung heißt zulässig gdw.alle $x_B\geq 0$
\end{itemize}
\textbf{Ecken}:\\
\begin{itemize}

\item Eine Ecke ist ein Punkt in G wenn er nicht als Mittelpunkt zweier anderer Punkte in G dargestellt werden kann
\item $x = \frac{1}{2} (x_1 + x_2 ), x_1 , x_2 \in G \Rightarrow x=x_1 = x_2$
\item Zu jeder Ecke gibt es mindestens eine zulässige Basislösung
\item degenerierte Ecke: Ecke mit mehreren zulässigen Basislösungen
\item ist eine Wert in der Basis 0 so ist die Ecke degeneriert
\item nicht-degenerierte Ecke: Ecke mit genau einer zulässigen Basislösung
\item in G gibt es mindestens eine aber maximal endlich viele Ecken
\end{itemize}
Es reicht bei einer linearen OP die Ecken nach der optimalen Lösung zu durchsuchen, da die Ecken die zulässigen Basislösungen sind\\
\hspace{1pt}\\
Eine Basislösung löst das OP gdw.
\begin{itemize}

\item Die Koeffizienten der Nichtbasis minus der Koeffizienten der Basis mal der Basismatrix mal der Nichtbasismatrix größer gleich 0 sind
\item $c_{N}^\intercal - c_{B}^\intercal (A_B)^{-1} A_N \geq 0$
\end{itemize}


\subsection{Primaler Simplex}
Besteht aus 2 Phasen:\\
1. Finden einer zul. Lösung
2. Finden der optimalen Lösung
\begin{itemize}
\item finden einer optimalen Lösung durch Wahl einer zulässigen Richtung mit maximaler Schrittweite $\rightarrow$ Verringerung des Zielfunktionswertes
\item x ist Optimal wenn alle $q \geq 0, q = c^\intercal_B (A_B )^{-1}b$
\item alle p müssen größer gleich 0 sein für eine Zulässigkeit
\item Solange ein Element in q < 0 gehe vom Punkt $x = (p^\intercal ,0^\intercal)^\intercal)$ in Richtung d mit $d_i= \begin{cases}
P_{i\tau}, & \text{wenn i in der Basisindexmenge liegt} \\
1, & \text{i = $\tau$ ist} \\
0, & \text{i in der Nichtbasisindexmenge ohne $\tau$ liegt}
\end{cases}$
\item $\tau$ sei der Index eines in q liegenden Elements kleiner 0
\item die maximale Schrittweite ergibt sich aus $\bar{t}(x, d) := min\{\frac{p_i}{P_{i\tau}}:P_{i\tau}<0, i\in I_B\}$
\item ist die maximale Schrittweite unendlich, gibt es keine Lösung da die ZF unbeschränkt ist
\item für entartete Ecken kann man eine Schrittweite von 0 erhalten
\end{itemize}

\textbf{Entscheidbarkeit}
\begin{itemize}
\item alle ZF-Koeffizienten sind größer gleich 0, d.h. es ist eine optimale Lösung
\item min ein Eintrag in q ist negativ, alle Einträger in P sind jedoch größer gleich 0, d.h. die ZF ist unbeschränkt und das Problem ist nicht lösbar
\item liegt keiner der beiden Fälle vor ist das Tableau nicht-entscheidbar, d.h. es muss ein Austauschschritt durchgeführt werden
\end{itemize}

\textbf{Simplex-Rechenregeln}
\begin{itemize}
\item Pivotspalte: Zeile mit kleinstem negativen ZF-Koeffizient
\item Pivotzeile: $-\frac{Eintrag rechte Seite}{Eintrag Pivotspale}$ für alle negativen Elemente der Pivotspalte, davon den kleinen Wert wählen
\item Pivorelement: Kreuzung aus PS und PZ
\item $PE_{neu} = \frac{1}{PE_{alt}}$
\item $PS_{neu} = \frac{PS_{alt}}{PE_{alt}}$
\item $PZ_{neu} = -\frac{PZ_{alt}}{PE_{alt}}$
\item $Rest_{neu} = Rest_{alt}-\frac{PS_{alt} PZ_{alt}}{PE_{alt}}$
\end{itemize}
Note: ist ein Element in q negativ, aber in der zugehörigen Spalte in A kein negatives Element, so ist die ZF unbeschränkt, die Abstiegsrichtung ist dabei $(x_i, s_i)^\intercal$ wobei das negative Element 1 gesetzt wird\\
\hspace{1pt}

\textbf{Finden einer zul. Ausgangslösung (Hilfszielfunktionsmethode)}\\
Sei das Problem $z=c^\intercal x \rightarrow min, \hspace{5pt} Ax = b, x \in \mathbb{R}_{+}^n$ gegeben
\begin{itemize}
\item Definiere Hilfsfunktion $h = e^\intercal y \rightarrow y$ mit $y+Ax=b, x\in \mathbb{R}_{+}^{n}, y \in \mathbb{R}_{+}^{m}$
\item e ist ein Vektor aus Einsen in der Dimension von y $e=(1,....1)^\intercal \in \mathbb{R^m}$
\item Erste Basislösung in Tableau mit $P = -A$, $p = b$, $q = -e^\intercal A$ und $ZFW = e^\intercal b$
\end{itemize}

\subsection{Dualer Simplex}

\begin{itemize}
\item Das Verfahren startet mit einem unzulässigen Punkt (anders als primaler Simplex)
\item die Folge der ZFW ist nicht monoton fallend (anderes als primaler Simplex)
\item Wird eine zul. Basislösung gefunden, wird diese auch optimal
\item Ist optimal wenn $p \geq 0, q \geq 0$
\item zulässig wenn alle Elemente in q größer gleich 0
\item p darf anderes als beim primalen Simplex Elemente kleiner 0 enthalten, sind alle Elemente in p > 0 ist es optimal
\end{itemize}

\textbf{Entscheidbarkeit}
\begin{itemize}
\item die rechte Seite sind alle größer gleich 0 $p \geq 0$, d.h. das Tableau ist optimal
\item eine Zeile in p ist negativ und alle Einträge in $p \leq 0$, d.h. das Problem ist nicht lösbar, der zulässige Bereich ist leer
\item trifft keine von beiden Aussagen zu ist das Tableau nicht-entscheidbar, d.h. es muss ein Austauschschritt durchgeführt werden
\end{itemize}

\textbf{Simplex-Rechenregeln}
\begin{itemize}
\item Pivotzeile: kleinster negativer Wert aus p
\item Pivotspalte: bilde $\frac{Eintrag ZF (q)}{Eintrag PZ}$ für alle positiven Werte der PZ, wähle den kleinsten Wert
\item Pivorelement: Kreuzung aus PZ und PS
\item $PE_{neu} = \frac{1}{PE_{alt}}$
\item $PS_{neu} = \frac{PS_{alt}}{PE_{alt}}$
\item $PZ_{neu} = -\frac{PZ_{alt}}{PE_{alt}}$
\item $Rest_{neu} = Rest_{alt}-\frac{PS_{alt} PZ_{alt}}{PE_{alt}}$
\end{itemize}

\paragraph{Dualität} 
\flushleft
\begin{itemize}
\item Sei (P) eine OA mit $z = c^\intercal x \rightarrow min, \hspace{5pt} Ax \leq b$
\item Sei (D) eine OA mit $z_D = -b^\intercal u \rightarrow max, \hspace{5pt}, A^\intercal u \geq -c$
\item um also ein primales in ein duales Problem umzuwandeln: Transponiere das vorhergehende Tableau, $p \leftrightarrow q^\intercal, A \leftrightarrow A^\intercal$
\end{itemize}
\hspace{1pt} \\
Dann ist (D) die duale OA zu (P)



\subsection{Branch and Bound}
\textbf{Vorgehensweise}:
\begin{itemize}
\item Benutzen einer Relaxation der zu optimierenden Funktion f durch Erweiterung des zulässigen Bereiches
\item Zerlegen des zulässigen Bereiches in Teilmengen $\rightarrow$ Zerlegung in Teilprobleme
\item Finden einer unteren Schranke $b(P_i)$ für jedes Teilproblem
\item für die untere Schranke muss gelten:
\begin{itemize}
\item muss kleiner gleich allen Funktionswerten von f(x) sein im Teilbereich
\item $b(P_i) \leq min\{f(x): x\in D \cap E_i \}$
\item die untere Schranke ist gleich dem Optima von f wenn das Teilproblem nur das Optima enthält
\item $b(P_i) = f(\hat{x}\text{, falls } D \cap E_i = \{\hat{x}\}$
\item die untere Schranke eines Teilbereiches $E_i$ ist kleiner gleich der eines Teilbereiches $E_j$ wenn $E_j$ Teil von $E_i$ ist
\item $b(P_i) \leq b(P_j)\text{, falls } E_j \subset E_i$
\end{itemize}
\end{itemize}
\textbf{Algorithmus}:
\flushleft Sei R Menge der noch zu bearbeitenden Teilprobleme\\
Sei z der ZF Wert der bisher besten gefundenen Lösung
\paragraph{Initialisierung}
\begin{itemize}
\item berechne untere Schranke
\item ist ein x für f(x) bekannt was gleich dieser Schranke ist, STOP: x löst P
\item Setze $R=\{P_0\}, z:=+\infty$ oder wenn ein x bekannt ist $z:=f(x)$
\end{itemize}
\paragraph{Abbruchtest}
\begin{itemize}
\item ist $R=\emptyset$ STOP
\item ist $z=+\infty$, dann ist P nicht lösbar
\item andernfalls ist x die Lösung von P
\end{itemize}
\paragraph{Strategie}
\begin{itemize}
\item wähle ein $P_i \in R$ und entferne dieses aus R ($R := R\setminus \{P_i\}$)
\item verschiedene Auswahlstrategien sind möglich z.B. Minimalsuche, DFS, BFS
\end{itemize}
\paragraph{Branch}
\begin{itemize}
\item Zerlegung von $P_i$ in Teilprobleme
\item setze j:=1
\end{itemize}
\paragraph{Bound}
\begin{itemize}
\item berechne untere Schranke für das erste Teilproblem ($b(P_{ij}$)
\item wird dabei ein x mit f(x) mit einem kleineren ZF z, setze z auf den neuen Wert
\item ist die untere Schranke kleiner als der ZF z dann füge das Teilproblem R hinzu ($R:=R \cup \{P_{ij}\}$)
\item wdh. für jedes Teilproblem den Bound Schritt
\item entferne alle Teilprobleme welche eine größere oder gleiche untere Schranke wie z haben ($R:=R \setminus \{P_k\}, b(P_k) \geq z$)
\item gehe zu Initialisierung
\end{itemize}

\section{Optimierung auf Graphen}
\subsection{Graphen}
Ein Graph ist:
\begin{itemize}
\item Menge an Tupeln $G=(V, E)$
\item V... Knotenmenge
\item E... Kantenmenge
\item sind die Kanten gerichtet (Bogen), so ist E ein Tupel aus Knoten (u,v), (von u nach v)
\item Zwei Knoten heißen adjazent, wenn sie durch eine Kannte verbunden sind
\item $N(v)$ sei die Menge der Nachbarknoten von v
\item ein Knoten ohne Nachbarn heißt isoliert
\item $\delta (v) := |\{u \in V: (u, v) \in E\}|$, ist der Knotengrad, also die Anzahl der Nachbarn
\item Sei $\delta^+(v)$ der Ausgangsgrad und $\delta^-(v)$ der Eingangsgrad eines Knoten v
\item Ein Knoten ohne Vorgänger heißt Quelle
\item Ein Knoten ohne Nachfolger heißt Senke
\item ein Zyklus ist eine Folge von Kanten, wobei mindestens der Start und Endpunkt der Selbe sind
\item ein Kreis ist eine Folge von Kanten, wobei nur der Start und Endpunkt der Selbe sind
\end{itemize}
Man unterscheidet folgende Graphen:
\begin{itemize}
\item schlichter Graph: keine Schleifen oder Mehrfachkanten
\item vollständiger Graph: besitzt alle möglichen Kannten
\item gewichteter Graph: jede Kannte $e \in E$ besitzt ein Gewicht $c(e)$
\end{itemize}

\subsection{Minimum Spanning Tree}
Ein Spannbaum/Gerüst ist:
\begin{itemize}
\item eine Teilmenge T eines Graphen G
\item G ist zusammenhängend, ungerichtet und gewichtet
\item der induzierte Subgraph von T soll kreisfrei sein
\item es ist also ein Graph welcher alle Knoten eines Graphen kreisfrei verbindet
\end{itemize}
\textbf{Ziel}:\\
Finde T mit kleinstmöglichem Gesamtgewicht $\sum_{e\in T} c(e)$

\subsubsection{Kruskal}
\begin{itemize}
\item Initialisiere eine leere Kantenmenge T
\item füge der Kantenmenge den kürzesten (von den Kosten geringsten) Weg hinzu
\item überprüfe ob T nun einen Kreis enthält
\begin{itemize}
\item wenn Ja, entferne diesen Weg wieder und fahre mit den nächst größeren fort
\item wenn Nein, fahre mit den nächst größeren Weg fort
\end{itemize}
\end{itemize}

\subsection{Max Flow}
Problem:
\begin{itemize}
\item gerichteter Graph G mit Kapazitäten k an allen Kanten $G = (V, E, k)$
\item es gibt eine Quelle und eine Senke
\item eine Funktion $x: E \rightarrow \mathbb{R}$ heißt Fluss, wenn eingehende "Flüssigkeit" gleich der ausgehenden "Flüssigkeit" ist, für alle Knoten
\item Es gilt Flusserhaltung, d.h. Die Menge welche an der Quelle austritt, muss in der Senke eingehen
\end{itemize}
\textbf{Ziel}:
Finde einen Graphen G mit einem Fluss welcher die maximale Flussstärke besitzt

\subsubsection{Residualgraph (Restkapazitätsgraph)}
\begin{itemize}
\item r(e) ist die Restkapazität (Residuum) einer Kannte, Kapazität der Kannte minus ausgehender Fluss plus eingehende Flüsse
\item $r(e) = k_e - x_e + x_{(v, u)}$
\item E(x) sei die Menge aller Kanten mit positiver Restkapazität
\item $E(x) = \{e \in E | r(e) >0 \}$
\item Sei G(x) der Residualgraph
\item $G(x) := (V, E(x), r(e)_{e\in E(x)})$
\end{itemize}

\subsubsection{Ford-Fulkerson}
Der Ablauf ist wie folgt:
\begin{itemize}
\item Setzte alle Flüsse im Graphen auf 0
\item X: Erstelle den Residualgraphen
\item gibt es keinen Weg von Quelle nach Senke mit positiver Flussstärke, dann STOP
\item sonnst, finde einen Weg von Quelle nach Senke mit maximaler Flussstärke f' entlang des Weges, 
\item aktualisiere dabei $x_e = x_e + x'_e$ für alle Kannten mit positivem Fluss
\item gehe zu X
\end{itemize}
Note: bei irrationalen Zahlen muss der Algorithmus nicht terminieren\\
Note: der Algorithmus verbessert die Lösung pro Iteration um min 1

\section{Auswahl an OP}

\subsection{Transportoptimierung}
gegeben ist:
\begin{itemize}
\item Erzeuger $i \in I = \{|I|\}$
\item Verbraucher $j \in J = \{|J|\}$
\item Kosten $c_{ij}$ für 1 Einheit von i nach j
\item Vorrat $a_i > 0$
\item Bedarf $b_j > 0$
\end{itemize}
\textbf{Ziel}: minimiere die Gesamtkosten

Sei  $x_{ij} \in \mathbb{R}_+$ Transportmenge von i nach j\\

\flushleft Constraints:
\flushleft gelieferte Einheiten zum Verbraucher dürfen dessen Bedarf nicht überschreiten
\[ \sum_{i\in I} x_{ij}\hspace{25pt} , j \in J  \]\\

\flushleft vom Vorrat verbrauchte Einheiten dürfen nicht größer als der Vorrat sein
\[ \sum_{j\in J} x_{ij}\hspace{25pt} , i \in I   \]\\

\flushleft es gibt nur positive Transportmengen
\[ x_{ij} \in \mathbb{R}_{+}\hspace{25pt} , (i,j) \in I \times J \]\\

Zielfunktion:
Kosten mal transportierte Einheiten für alle Transportwege\\
\[z= \sum_{i\in I} \sum_{j\in J} c_{ij}x_{ij} \rightarrow min \]

\subsection{Rucksackproblem}
gegeben ist:
\begin{itemize}
\item Rucksack mit Kapazität $b \in \mathbb{Z}_+$
\item m zu verpackende Teile
\item Gewicht eines Teiles $a_i \in \mathbb{Z}_+$
\item Nutzwert eines Teiles $c_i \in \mathbb{Z}_+$
\item $i \in I = \{|I|\}$
\end{itemize}
\textbf{Ziel}: wähle Teile für maximalen Nutzwert\\
\hspace{1pt}\\

Sei $x_i = \begin{cases}
0, & \text{Teil i wird nicht mitgenommen} \\
1, & \text{Teil i wird mitgenommen}
\end{cases}$

\flushleft Constraints:
\flushleft dürfen nicht mehr Teile mitgenommen werden als in den Rucksack passen
\[ \sum_{i\in I} a_{i}x_{i} \leq b \]\\

0-1 Rucksackproblem
\flushleft ein Teil kann nur mitgenommen werden oder nicht
\[ x_i \in \mathbb{B}\hspace{25pt} , i\in I \]\\

klassisches Rucksackproblem
\flushleft ein Teil kann nur ganzzahlig positiv oft mitgenommen werden
\[ x_i \in \mathbb{Z}_+\hspace{25pt} , i\in I \]\\

Zielfunktion: Wertsumme der mitgenommen Teile maximieren
\[ \sum_{i\in I}c_{i}x_{i} \rightarrow max  \]

\subsection{Bin Packing Problem}
gegeben ist:
\begin{itemize}
\item unendlich viele Behälter mit Kapazität $L \in \mathbb{N}$
\item Teile $b_i, i \in I = \{|I|\}$
\item Gewicht eines Teils $l_i, i \in I = \{|I|\}$
\item Packung $a_i, i \in I = \{|I|\}$
\end{itemize}
\textbf{Ziel}: minimal benötige Behälteranzahl\\

\flushleft Constraints:
\flushleft Gewicht einer Packung darf die Kapazität des Behälters nicht überschreiten
\[ \sum_{i\in I} l_{i}a_{i} \leq L \]\\

\subsection{Standortplanung}
gegeben ist:
\begin{itemize}
\item Kunden $k \in K = \{|K|\}$
\item Standorte $s \in S = \{|S|\}$
\end{itemize}
\textbf{Ziel}: 
\begin{itemize}
\item minimale Baukosten
\item minimale Laufkosten
\item Bedarf aller Kunden Bedienen
\end{itemize}
Seien: \begin{itemize}
\item $d_{ks} \in \mathbb{R}$ Kosten um Kunde k am Standort s zu bedienen
\item $c_s \in  \mathbb{R}$ Baukosten für den Standort
\item $x_s = \begin{cases}
0, & \text{Standort wird nicht genommen} \\
1, & \text{Standort wird genommen}
\end{cases}$
\item $y_{ks}$ Teil des Bedarfs des Kunden k welcher durch den Standort s gedeckt wird
\end{itemize}

\flushleft Constraints:
\flushleft der Gesamtbedarf des Kunden muss gedeckt sein
\[ \sum_{k\in K}y_{ks} = 1\hspace{25pt} ,k\in K \]\\

\flushleft Ein Standort kann maximal den gesamten Bedarf eines Kunden decken
\[y_{ks} \leq x_s\hspace{25pt} , (k,s) \in K\times S  \]\\

\flushleft Der Bedarf kann nicht negativ sein (und nicht größer als 1 durch constraint 1)
\[ y_{ks} \in \mathbb{R}_+\hspace{25pt} , (k,s) \in K\times S \]\\

\flushleft Ein Standort kann nur liefern oder nicht liefern
\[ x_s \in \mathbb{B}\hspace{25pt} , s\in S \]\\

Zielfunktion:
\begin{itemize}
\item Baukosten der Standorte (erster Summand) Summe der zu bauenden Standorte (selektiert durch $x_s = 1$)
\item Laufkosten der Standorte (zweiter Summand) Summe Bedienkosten des Kunden mal Anteil des von Kunden geforderten gesamt Bedarfs
\end{itemize}
\[ \sum_{s\in S}c_{s}x_{s} + \sum_{k\in K} \sum_{s\in S} d_{ks}y_{ks} \rightarrow min \]


\section{How to solve}

\subsection{Ermitteln ob eine Richtung zulässig im Punkt x ist}
\textbf{gegeben}:
\begin{itemize}
\item Punkt x
\item Menge G mit Restringtionen a
\item Richtung d
\end{itemize}
\paragraph{Ermittle alle in x aktiven Restringtionen $I_0$\\}

Errechne $I(x,d) := \{i \in I: a_{i}^\intercal d \leq 0\}$\\

\paragraph{Überprüfe ob die Richtung zulässig ist\\}
Für alle $i \in I$ (aus step 1) rechne $a_{i}^\intercal d$
\begin{itemize}
\item wenn > 0: keine zulässige Richtung
\item wenn $\leq$ 0: zulässige Richtung
\end{itemize}

\subsection{Ermitteln der Intervall von \tsnake t}
\textbf{gegeben}:
\begin{itemize}
\item Punkt x
\item Menge G mit Restringtionen ax = b
\item Richtung d
\end{itemize} 
\paragraph{Errechne M\\}

Berechne $M = A^\intercal (x + td)$ A ist die Matrix welche sich aus allen $a_i$ ergibt

\paragraph{Ermittle untere Grenze für t\\}
\begin{itemize}
\item Vergleiche M und b zeilenweise mit $\forall t \in \mathbb{R}^+$
\item Maximale Schrittweite ist das Maximum für t wobei $m_i \leq b_i$ noch gilt
\item Oder: Stelle Zeile $m_i \leq b_i$ nach t um = untere Schranke $t \in [s, \inf)$
\end{itemize}

\paragraph{Ermittle obere Grenze für t (\tsnake t)\\}
\begin{itemize}
\item Errechne $(x+td)$
\item Löse $(x+td) \geq \vec{0}$ nach t auf für jede Zeile
\item der kleine erhaltene Wert ist die obere Grenze \tsnake t
\end{itemize}

\subsection{Ermitteln des Kegels der zulässigen Richtung}
\textbf{gegeben}:
\begin{itemize}
\item Restringtionsmatrix A
\item Absolutteil der Restingtionen b
\item Punkte $x^i$
\end{itemize}
Berechne Z(x)
\begin{itemize}
\item berechne $A*x^i \leq b, i \in I$
\item überprüfe Spaltenweise ob das Ergebnis kleiner gleich dem Vektor b ist
\item Folge Fälle können für $a_i * x^i$ und $b_i$ eintreten:
\begin{itemize}
\item alle Elemente sind kleiner: es handelt sich um einen inneren Punk, alle Richtungen sind zulässig
\item alle Elemente sind größer: es gibt keine valide Richtung, es handelt sich um einen äußeren Punkt
\item eine/alle Zeile(n) erfüllt(en) die Bedingung mit Gleichheit: es sind alle Richtung valide, welche die Restringtion der entsprechenden Zeile(n) verletzen
\end{itemize}
\end{itemize}

\subsection{Finden einer zulässigen Anfangslösung für Transportalgorithmus}
\subsubsection{NW-Regel}
\begin{itemize}
\item Starte oben links und weise maximale Kapazität zu
\item wenn alle zu vergebenden Einheiten der Zeile oder benötigten Einheiten der Spalte verbraucht sind, setzte entsprechende Spalte (oder Zeile) auf 0
\item bewege nur nach rechts oder unten, gehe nie nach links oder oben
\end{itemize}
\subsubsection{Minimale-Kosten-Regel}
\begin{itemize}
\item suche kleinen Kostenwert
\item weise dort maximal mögliche Kapazität zu
\item gehe zum nächst größerem Wert
\item ist Verbraucher/Erzeuger leer, fülle Zeile (oder Spalte) mit 0
\end{itemize}

\subsection{Lösen des Transportplans}
Anfangend mit einer zulässigen Lösung:
\begin{itemize}
\item berechne für alle Basiszellen (alles wo keine 0 drin steht) $u_i + v_k = c_{ik}$
\item berechne für den Rest $w_{ik} = c_{ik} - u_i - v_k$, sind alle $w_{ik} \geq 0$ ist der Plan optimal
\item wähle das kleinste errechnete Element der Nichtbasiszellen
\item bilde von diesem Element aus einen Zyklus durch 3 andere Basiszellen
\item markiere die Nichtbasiszelle des Zyklus mit einem (+)
\item markiere nun abwechselnd alle Zellen des Zyklus mit (-) und (+)
\item bilde $\delta$, dass Minimum der (-) Zellen
\item rechne je nach Markierung Zelle +/- $\delta$
\item dadurch wird ein Element des Zyklus 0, dieses wird Teil der Nichtbasis, das andere nun größer als 0 Element wird dafür Teil der Basis
\end{itemize}

\subsection{Ganzzahlige Optimierung mit Simplex}
\begin{itemize}
\item löse Simplextableau ganz normal mit primalen Simplex
\item für alle nicht ganzzahligen erzeuge zwei Teilprobleme, einmal auf bzw. abgerundet
\item führe neue Schlupfvariable ein und setze bekannte Werte ein
\item überprüfe ob zulässiger Bereich dadurch leer wird, wenn ja STOP
\item erstelle neues Tableau und löse mit dualem Simplex
\item wdh. bis alle Werte ganzzahlig
\item bei mehreren möglichen Verzweigungen schauen welche in einem kleineren ZF resultiert
\end{itemize}

\subsection{Branch and Bound mit TSP}
\textbf{Zeilen/Spalten Reduktion}
\begin{itemize}
\item Ziehe von jedem Element der Zeile/Spalte den Wert des kleinsten Elementes der Zeile/Spalte ab
\item Addiere die Abgezogenen Werte, dies ist die untere Schranke b
\item je nachdem ob erst Zeilen oder Spalten reduziert werden kann eine andere untere Schranke entstehen
\end{itemize}
\textbf{Aufstellen der Kostenmatrix}
\begin{itemize}
\item Errechne Kosten für alle Nulleinträge
\item Kosten ergeben sich aus dem kleinem Wert der Spalte und dem kleinem Wert der Zeile des Elementes, das Element selbst zählt nicht
\item $w_{ik} = \min d_{ik} + \min d_{pk}, q \neq k, p \neq i$
\end{itemize}
\textbf{Branch}
Verzweige für das Element mit den größten Kosten, es gibt folgende Optionen
\begin{itemize}
\item $x_{ik} = 0$ benutze diesen Weg nicht
\begin{itemize}
\item setze das Element auf $\infty$
\item führe Spalten/Zeilenreduktion aus
\item errechne neue Schranke
\item Ist die Schranke größer als der bekannte ZF muss diese Verzweigung nicht weiter untersucht werden
\end{itemize}
\item $x_{ik} = 1$ benutze diesen Weg
\begin{itemize}
\item setze alle Elemente in der Zeile und Spalte, außer das Element selbst auf $\infty$
\item setze ebenfalls das Element welches eine Subtour erzeugt auf $\infty$ z.B. bei $x_{32}$ setze $x_{23}$ auf $\infty$
\item führe Zeilen/Spaltenreduktion aus und berechne neue untere Schranke
\item Ist die Schranke größer als der bekannte ZF muss diese Verzweigung nicht weiter untersucht werden
\end{itemize}
\item führe diese Verzweigungen solange aus bis der ZFW mit der unteren Schranke erreicht wird, oder feststeht das es nicht besser geht
\end{itemize}


\subsection{ganzzahlige Lösung mittel Gomory-Schnitt}
\begin{itemize}
\item ersetze eine Zeile mit ungerader Lösung im gelösten Simplextableaus mit einer Gomory Zeile
\item für die Einträge in A, runde die Einträge auf und ziehe davon den ungerundeten Einträg ab
\item für den Eintrag in p, ziehe den abgerundeten Wert vom ungerundeten ab
\item \[ p_i - \lfloor p_i \rfloor \leq \sum_{j\in N} (\lceil P_{ij} \rceil - P_{ij})x_j \]
\item führe Schlupfvariable ein, stelle nach dieser um und füge sie dem Tableau hinzu
\item löse das neue Simplextableau
\end{itemize}
wdh. bis alle Variablen ganzzahlig sind


\subsection{Erstellung eines Residualgraphen}
\begin{itemize}
\item für jede benutzte Kapazitätseinheit einer kannte geht im Residualgraph eine Kannte entgegengesetzt dazu
\item als Kapazität steht dann das Residuum, also das was zuvor noch übrig war
\item bei einer zuvor voll belasteten Kante ist es also (0/(Kapazität)) sonst (0/(Kapazität - untere Schranke))
\item war die Kante zuvor nicht voll ausgelastet, so bleibt eine Kante in die ursprüngliche Richtung bestehen
\item die Kapazität davon ist dann (0/(zuvor ungenutzte Kapazität))
\end{itemize}

\section{Beispiele}
\subsection{Welche Richtungen d sind vom Punkt x aus zulässig, was ist die maximale Schrittweite}

\textbf{gegeben}:
\begin{itemize}
\item Punkt $x = (1, 1, 1)^\intercal$
\item Richtungen $d^{i} = \begin{pmatrix} 1 & -1 \\ 1 & -2 \\ 1 & -1 \end{pmatrix}$
\item Restringtionsmatrix  $A = \begin{pmatrix} 1 & 3 \\ 2 & 1 \\ 1 & 1 \end{pmatrix}$
\end{itemize}
\textbf{Lösung}:\\
Für $d_1$\\
$A^\intercal d_1 \leq \vec{0} = (4, 5)^\intercal \leq (0 0)^\intercal$ ist nicht erfüllt, keine zulässige Richtung\\

Für $d_2$\\
$A^\intercal d_2 \leq \vec{0} = (-6, -6)^\intercal \leq (0 0)^\intercal$ ist erfüllt, d2 ist eine zulässige Richtung\\

$M = A^\intercal (x+dt) = \begin{pmatrix} 1 & 2 & 1 \\ 3 & 1 & 1 \end{pmatrix} (\begin{pmatrix}1 \\ 1 \\ 1 \end{pmatrix}+ \begin{pmatrix} -1 \\ -2 \\ -1 \end{pmatrix}t)
= \begin{pmatrix} -4t + 4 \\ -6t + 5 \end{pmatrix}$\\

Stelle $m_i \leq b_i$ nach t um:\\
$-4t +4 \leq 4 \Leftrightarrow t \geq 0$\\
$-6t +5 \leq 6 \Leftrightarrow t \geq \frac{-1}{6}$\\
\hspace{1pt}\\
$\Rightarrow t \in [-1/6, \inf)$

Stelle $(x+dt) \geq \vec{0}$ nach t um:\\ 
$t \leq 1$\\
$t \leq \frac{1}{2}$\\
$t \leq 1$\\
$\Rightarrow$ nehme das Minimum
\tsnake t = 1/2

\subsection{finden des Kegels der zulässigen Richtung}
\textbf{gegeben}:
\begin{itemize}
\item ZF: $ z = -x_1 - x_2 \rightarrow min$
\item Restingtionen:
\begin{itemize}
\item $x_1 + 2x_2 \leq 8$
\item $x_1 + x_2 \leq 5$
\item $x_1,x_2 \geq 0$
\end{itemize} 
\item Punkte: $x^1 = (1, 1)^\intercal, x^2 = (3, 3)^\intercal, x^3 = (4, 1)^\intercal, x^4 = (2, 3)^\intercal, x^5 = (0, 0)^\intercal$
\end{itemize}
\textbf{Lösung}:
\begin{itemize}
\item schreibe die Punke in Matrix $X = \begin{pmatrix}1 & 3 & 4 & 2 & 0 \\
 1 & 3 & 1 & 3 & 0
\end{pmatrix} $
\item stelle Restringtionsmatrix A und Absolutvektor b auf: $A = \begin{pmatrix}
1 & 2 \\ 1 & 1 \\ 1 & 1
\end{pmatrix}, b = \begin{pmatrix}
8 \\ 5 \\ 0
\end{pmatrix}$

\item rechne und vergleiche Spaltenweise $A\cdot X \leq b$: $\begin{pmatrix}
3 & 9 & 6 & 8 & 0 \\ 2 & 6 & 5 & 5 & 0 \\ 2 & 6 & 5 & 5 & 0 \end{pmatrix} \leq \begin{pmatrix}
8 \\ 5 \\ 0
\end{pmatrix}$
\end{itemize}
\textbf{Ergebnis}:
\begin{itemize}
\item $x^1$ ist ein innerer Punk, da beide Koordinaten kleiner b: $Z(x^1) = \mathbb{R}^2$
\item $x^2$ ist außerhalb des zulässigen Bereichs, da beide Koordinaten größer b: $Z(x^2) = \emptyset$
\item $x^3$ erfüllt Zeile 2 mit Gleichheit, d.h. $Z(x^3) = \{d \in \mathbb{R}^2 | d_1 + d_2 \leq 0 \}$
\item $x^4$ erfüllt die ersten beiden Zeilen mit Gleichheit, d.h. $Z(x^4) = \{d \in \mathbb{R}^2 | d_1 +d_2 \leq 0 , d_1 + 2d_2 \leq 0 \}$
\item $x^5$ erfüllt die letzte Restingtion, d.h. $Z(x^5) = \{d \in \mathbb{R}^2 | d_1 \geq 0, d_2 \geq 0 \}$
\end{itemize}

\subsection{finden einer Ganzzahligen Lösung}
\textbf{gegeben}:
\begin{itemize}
\item $\bar{z} = -z = -7x_1 -2x_2 \rightarrow min$ mit,
\item $-x_1 + 2x_2 + s_3 = 4 \Rightarrow s_3 = x_1 -2x_2 +4$
\item $5x_1 + x_2 + s_4 = 20 \Rightarrow s_4 = -5x_1 -x_2 + 20$
\item $x_1, x_2, s_3, s_4 \geq 0$
\end{itemize}
\textbf{Schritt 1, Simplex}.
Lösung mit primalem Simplex ergibt
\begin{itemize}
\item $x_1 = \frac{36}{11}$
\item $x_2 = \frac{40}{11}$
\item $s_3, s_4 = 0$
\item $z = -\frac{332}{11}$
\end{itemize}
\textbf{Schritt 2, Branch}
Verzweige für $x_2$ da nicht ganzzahlig (und am wenigsten an einer ganzen Zahl dran)
\begin{itemize}
\item $x_2$ liegt zwischen 3 und 4
\item 1. TP $x_2 \geq 4 \Rightarrow s_3 = x_2 -4$
\item 2. TP $x_2 = \leq 3 \Rightarrow s_3 = 3 - x_2$
\item setzte $x_2$ in jeweiliges TP ein und überprüfe ob zul. Bereich leer ist
\item TP 1 ist leer da TP 1 < 0 für $s_4, s_3 \geq 0$, TP 2 ist nicht leer, verzweige
\item nehme neue Gleichung für TP 2 in Simplextableau auf ($ s_3 = \frac{1}{11} + \frac{5}{11} + -\frac{7}{11}$)
\end{itemize}
\textbf{Schritt 3, dualer Simplex}
Nach lösung mit dualem Simplex
\begin{itemize}
\item $x_2 = 3$
\item $x_1 = \frac{17}{5}$
\item $s_3 = \frac{7}{5}$
\end{itemize}
\textbf{Wdh Schritt 2 und 3 für $x_1 \geq 4$}
Nach Verzweigung und Simplex ist
\begin{itemize}
\item $x_2 = 0$
\item $x_1 = 4$
\item $x_3 = 8$
\item $s_2 = 3$
\end{itemize}
der ZFW beträgt bei dieser Wahl -28

\textbf{Wdh Schritt 2 und 3 für $x_1 \leq 3$}
Nach Verzweigung und Simplex ist
\begin{itemize}
\item $x_2 = 3$
\item $x_1 = 3$
\item $x_3 = 2$
\item $s_2 = 1$
\end{itemize}
der ZFW beträgt bei dieser Wahl -27\\

Die optimale Lösung ist $z_{max} = 28, x_1 = 4, x_2 = 0$



\end{document}