\documentclass[12pt,a4paper]{article}
\usepackage[legalpaper, landscape, margin=0cm]{geometry}
\usepackage[utf8]{inputenc}
\usepackage[german]{babel}
\usepackage[T1]{fontenc}
\usepackage{times}
\usepackage{amsmath}
\usepackage{amsfonts}
\usepackage{float}
\usepackage{stix}
\usepackage{amssymb}
\usepackage[shortlabels]{enumitem}
\usepackage{enumitem}
\usepackage{changepage}
\usepackage{paracol}
\usepackage{xcolor}
\setlength{\columnseprule}{0.4pt}
\newcommand{\blue}[1]{\textcolor{blue} {#1}}
\begin{document}

\begin{paracol}{3}
\switchcolumn[0]
\centering
\textbf{Reduktionen}
\flushleft
\begin{itemize}
\setlength\itemsep{-0.5em}
\item turing red.: $P \leq_T Q$ P kann mit Q als subroutine gelößt werden
\item poly. m-red.: $ P \leq_p Q \Leftrightarrow \forall w \in \Sigma^*: w\in P \leftrightarrow f(w) \in Q$
\item L ist NP-hard: $\forall L' \in $ NP $: L' \leq_p L$ (alle L' aus NP sind auf L p.m.r.)
\item L element NP: ex. ein poly. Verifikator (Cert. überprüfbar in P)
\item L ist NP-complete: $L \in $ NP und L NP-hard
\item Jede m-red ist eine t-red (aber nicht anders herum)\\ex. P s.d. $P \leq_T Q$ aber nicht $P \leq_m Q$
\end{itemize}
\hspace{1cm}
\switchcolumn[1]
\centering
\textbf{(Un)Entscheidbarkeit}
\flushleft
\begin{itemize}
\setlength\itemsep{-0.5em}
\item entscheidbar: ex. TM, die f.a. inputs $w (\in | \not\in) L$ hält
\item \blue{(co)}-semi-entscheidbar: ex. TM, die f.a. inputs $w \blue{(\not\in)} \in  L$ hält
\item $P \leq Q$, Q \blue{(un)}entscheidbar $\Rightarrow$ P \blue{(un)entscheidbar}
\item semi- + co-semi = entscheidbar = $L$ und $\bar{L}$ entscheidbar
\item $Phalt \leq P \land PHalt \leq \bar{P}\, \Rightarrow$ vollständig unentscheidbar
\item Satz v. Rice: E nicht-triviale Eigenschaft, TM M mit L=L(M)\\hat L die E ? $\Rightarrow$ Unentscheidbar
\end{itemize}
\switchcolumn[2]
\centering
\textbf{Beziehungen von Klassen}
\flushleft
\begin{itemize}
\setlength\itemsep{-0.5em}
\item L $\subseteq$ NL $\subseteq$ P $\subseteq$ NP $\subseteq$ PSpace $\subseteq$ NPSpace $\subseteq$ EXP $\subseteq$ NEXP
\item Satz von Savitch: NSpace(f(n)) $\subseteq$ DSpace(f(n)$^2$)
\item NL $\subsetneq$ PSpace, P $\subsetneq$ EXP, NP $\subsetneq$ NEXP (folgt auf THT/SHT)
\item Satz von Immerman, Szelepcsenyi: NL $=$ coNL
\item $L \in$ DTIME/DSPACE $\Leftrightarrow$ $\bar{L}\in $ DTIME/DSPACE
\item wenn NP $=$ coNP $\rightarrow$ P $=$ NP
\end{itemize}
\switchcolumn[0]*[\hrule width\textwidth]
\centering
\textbf{Halteproblem}
\flushleft
\begin{itemize}
\setlength\itemsep{-0.5em}
\item PHalt = geg. M, w hält M auf w ?
\item Assume: PHalt entscheidbar $\rightarrow$ ex. Entscheider H
\item Konstruiere d. Diagonalisierung TM D s.d.
\item -  prüfe ob Eingabe TM ist
\item -  Simuliere H auf <M,<M>> $\rightarrow$ hält M auf <M> ?
\item -  Ja $\rightarrow$ Endlosschleife, Nein $\rightarrow$ akzeptiere
\item \textit{Simuliere D auf <D>} $\rightarrow$ Wiederspruch
\end{itemize}
\switchcolumn[1]
\centering
\textbf{PCP}
\flushleft
\begin{itemize}
\setlength\itemsep{-0.5em}
\item Menge Dominosteine, ex. Anordnung s.d. oben = unten
\item $PHalt \leq mPCP \leq PCP$
\item mPCP: PCP aber mit festem Startstein
\item idee: Kodieren TM-lauf (config. seq) als seq. von Wortpaaren, s.d oben eine config zurücl liegt
\item ->  TM hält $\rightarrow$ Lauf endlich, ex. Lösung (+ vice versa)
\item MPCP hat lsg. $\rightarrow$ PCP hat lsg. + $\forall$ Sym. umgeben mit \#
\item PCP hat lsg. $\rightarrow$ mPCP hat lsg. + da PCP mit erstem Stein anfangen muss, da nur dieser oben = unten + \# weglassen
\end{itemize}
\switchcolumn[2]
\centering
\textbf{Unifikation}
\flushleft
\begin{itemize}
\setlength\itemsep{-0.5em}
\item Löschen: $\{x = x\} \rightarrow \{\}$ (gleiches weglassen)
\item Zerlegung: \\ $\{f(x_1, ..., x_n) = f(y_1, ..., y_n)\} \rightarrow \{x_1=y_1, ..., x_n=y_n\}$
\item Orientierung: $\{x=y\} \rightarrow \{y=x\}$ (vertauschen)
\item Eliminierung: $\{x=y\} \rightarrow \{x=y\} \cup G\{x\mapsto y\}$ (einsetzen)
\item $\sigma$ is min. so allgemein wie $\theta$, wenn $\exists \lambda: \sigma \circ \lambda = \theta$,
\item MGU: $\forall \theta: \sigma \preceq \theta$
\item alle MGU's sind bis auch Umbenennung der Vars gleich
\item gelößte Form: wenn RS nicht in LS vorkommt, RS ist Variable
\end{itemize}
\switchcolumn[0]*[\hrule width\textwidth]
\centering
\textbf{Prädikatenlogik Definitionen}
\flushleft
\begin{itemize}
\setlength\itemsep{-0.5em}
\item V...Variablen, C...Konstanten, P...Prädikatensymbole
\item Atom: $p(t_1,...,t_n)\;\quad p\in P,\; t_i\in V\cup C$ (t...Terme)
\item Formel: Induktiv, Atom ist Formel, $\lnot F, F\land G,..., \forall x.F$ auch
\item gebundene Variable: steht im Scope eines Quantors\\default: Variablen in Atomen sind frei
\item Interpretation $\mathcal{I}$: $\langle \Delta^\mathcal{I}, \cdot^\mathcal{I} \rangle$
\item -  $\Delta^\mathcal{I}$: Domäne/Grunmenge
\item -  $\cdot^\mathcal{I}$: Interpretationsfunktion
\item $\mathcal{I}$ mapped jede Konstante auf ein Domänenelement und jedes Prädikatensymbol auf eine Relation (Menge)
\item Bsp: $\Delta^\mathcal{I} = \mathbb{N}$, $c^\mathcal{I}=299.792.458$, $\leq_4^\mathcal{I} = \{0,1,2,3,4\}$,\\even$^\mathcal{I} = \{n|n \equiv 0 $ (mod 2)$\}$
\item Zuweisung: $\mathcal{Z}: V \rightarrow \Delta^\mathcal{I}$, $\mathcal{Z}[x\mapsto \delta]$, für $x\in V$, $\delta \in \Delta^\mathcal{I}$
\item Kompaktheitssatz:\\$|T|=\infty, T\models F, F\models G, G\subset T, |G|=\infty$
\end{itemize}
\switchcolumn[1]
\centering
\textbf{Formelformen}
\flushleft
\begin{itemize}
\setlength\itemsep{-0.5em}
\item geschlossene Formel/Satz: Formel ohne freie Variablen
\item negation normal Form (NNF): nur ($\lnot, \land, \lor$), $\lnot$ nur vor Atomen
\item bereinigte Form: keine Variable kommt gebunden und frei vor\\keine Variable wird von mehr als einem Quantor gebunden
\item Pränexform: Alle Quantoren stehen am Anfang
\item Skolemform: Pränexform + alle $\exists$ entfernt\\n-stelliges Funktionssymbol mit n = Anz. der $\forall$ davor
\item KNF: $(...\lor ... \lor ...) \land (...\lor ...) \land ...$
\item Klauselform: $(x\lor y)\land \neg z$ $\rightarrow$ $\{\{x, y\}, \{\neg z\}\}$
\item F in Pränex erfüllbar $\Leftrightarrow$ Skolemisierung erfüllbar
\item Skolemisierung is keine äquiv. Umf., nur erfüllbarkeitserhalte
\end{itemize}
\switchcolumn[2]
\centering
\textbf{Logigisches Schließen}
\flushleft
\begin{itemize}
\setlength\itemsep{-0.5em}
\item $\mathcal{I}$ is ein Modell für F wenn $\mathcal{I}$ F erfüllt ($\mathcal{I}\models F$)
\item $\mathcal{I}\models\mathcal{T}$ wenn $\forall F \in \mathcal{T}: \mathcal{I}\models F$
\item logische Konsequenz: $F\models G$, jedes Modell für F ist eins für G
\item Alle Tautologien $\models F$ sind äquivalent (ebenso für unerf. F)
\item $F \equiv G \Leftrightarrow F\models G \land G \models F$
\item Monotonie: Mehr Sätze $\Leftrightarrow$ weniger Modele, mehr Schlussfolgerungen\\$\rightarrow$ Tautologien sind in jeder $\mathcal{I}$ wahr, log. Kons. von allem\\$\rightarrow$ Unerf. F haben kein Model, haben alle Sätze als Kons.
\item Deduktionstheorem: $F \models G \rightarrow H \Leftrightarrow F \cup \{G\} \models H$
\item $F \land G \models H \Leftrightarrow F \models G \rightarrow H$
\item $F\equiv G \Leftrightarrow \models F \leftrightarrow G$
\item $T \models F \equiv T \cup \{\neg F\}$ unerf. ? $\equiv (\land_{G\in\mathcal{T}} G \rightarrow F$ Tautologie ?
\end{itemize}
\end{paracol}
\newpage
\begin{paracol}{3}
\switchcolumn[0]
\centering
\textbf{Herbrand}
\flushleft
\begin{itemize}
\setlength\itemsep{-0.5em}
\item Satz in Skolemform erf $\Leftrightarrow$ F hat HB-Model
\item Herbrand-Expansion:\\$HE(F) := \{G\{x_1\mapsto t_1, ..., x_n\mapsto t_n\}|t_1,...,t_n\in\Delta_F\}$\\z.B. $p(a,f(a,s,t))\lor q(b)) \,|\, s,t \in \Delta_G\}$
\item Satz v. Göd, Her, Sko: F in SkF erf. $\Leftrightarrow$ HE(F) aussagenlog. erf
\item Herbrand-Universum: für F, Menge aller var.freien Terme, die man mit Konst. und Funk. symbolen in $F \cup \{a\}$ bilden kann
\item Herbrand-Interpretation: $\Delta^\mathcal{I} = \Delta_F$, $\forall t\in \Delta_F: t^\mathcal{I}=t$
\end{itemize}
\switchcolumn[1]
\centering
\textbf{Clique NPC}
\flushleft
\begin{itemize}
\setlength\itemsep{-0.5em}
\item Clique $\in$ NP: Clique ist Cert.
\item Clique NP-hard: SAT $\leq_p$ Clique
\item $G_F$ hat Clique der Größe l$\Leftrightarrow$ F ist erf.
\item $F = \big( (L_1^1\lor...\lor L_{n_1}^1)\land ...\land (L_1^l\lor ...\lor L_{n_l}^l)\big)$
\item Knoten: $V = \{(L_j^i)\,|\, i\in \{1...l\},j\in\{1,...,n_i\}\}$
\item Kanten: $E = \{(L,i)-(L',j) \,|\, i\neq j, L\land L' $ erf. $\}$
\item ez auf unabh. Menge reduz. $\rightarrow$ Komplement
\end{itemize}
\switchcolumn[2]
\centering
\textbf{Basic Logikregeln}
\flushleft
\begin{itemize}
\setlength\itemsep{-0.5em}
\item De-Morgan: $\neg (A \land  B) \equiv \neg A \lor \neg B$
\item De-Morgan: $\neg (A \lor B) \equiv \neg A \land \neg B$
\item $A \rightarrow B \equiv \neg A \lor B$
\item $A \leftrightarrow B \equiv A \land B \lor \neg A \land \neg B$
\item $\neg \neg A \equiv A$
\end{itemize}
\switchcolumn[0]*[\hrule width\textwidth]
\centering
\textbf{Teilmengen Summe NPC}
\flushleft
\begin{itemize}
\setlength\itemsep{-0.5em}
\item $F = (C_1 \land ... \land C_n)$
\item $v(t_i) := a_1...a_nc_1...c_k$
\item $\displaystyle a_j = \begin{cases}1, i=j\\0,  i\neq j\end{cases},\quad c_j = \begin{cases}1, p_i \in C_j\\0, else\end{cases}$
\item $v(f_i):=a_1...a_nc_1...c_k$
\item $\displaystyle a_j = \begin{cases}1, i=j\\0, i\neq j \end{cases},\quad c_j = \begin{cases}1, \neg p \in C_j\\0 else \end{cases}$
\item für $r := |C_i| -1$, def: $m_{i, 1},...,m_{i,r}$
\item $v(m_{i,j}) := c_i...c_k$
\item $\displaystyle c_l = \begin{cases}1, l=i\\0, l\neq i \end{cases}$
\item $\displaystyle S := \big\{t_i,f_i \, \big\vert \, 1 \leq i \leq n \big\} \cup \big\{ m_{i,j} \, \big\vert\, 1\leq i\leq k, 1\leq j \leq |C_i|-1\big\}$
\item $z := a_1...a_nc_1...c_k$ mit $a_i := 1, c_i = |C_i|$
\end{itemize}
\switchcolumn[1]
\centering
\textbf{PCP Steine}
\flushleft
\begin{itemize}
\setlength\itemsep{-0.5em}
\item $$
\begin{cases}
\left[ {qa \atop bp} \right] \rightarrow \delta(q,a) = \langle p,b,R \rangle,\\
\left[ {cqa \atop pcb} \right] \rightarrow \delta(q,a) = \langle p,b,L \rangle \, \text{und $c \in \Gamma$ beliebig},\\
\left[ {qa \atop pb} \right] \rightarrow \delta(q,a) = \langle q,b,N \rangle\\
\end{cases}
$$
\item $$
\begin{cases}
\left[ {\# qa \atop \# bp} \right] \rightarrow \delta(q, a) = \langle p, b, L \rangle \, \text{anstoßen am linken Rand},\\
\left[ {q \# \atop q \mathvisiblespace \# } \right] \rightarrow \forall q \in Q \, \text{unendlich nach rechts}\\
\end{cases}
$$
\item Kopierregel:
$\left[ {x \atop x} \right] \rightarrow \forall x \in \Gamma \cup \{ \# \}$
\end{itemize}
\switchcolumn[2]
\centering
\textbf{Datalog}
\flushleft
\begin{itemize}
\setlength\itemsep{-0.5em}
\item Konsequenzoperator: $T_p$
\item $T_p^0 = \emptyset$, if $T_p^i = T_p^{i+1} \rightarrow T_p^\infty$
\item n-spaltige Tabelle wird zu n-stelligem Prädikat
\item Schlussfolgerung $P\models F$ in Datalog in ExpTime
\item für jeden Fakt $F\in T_p^\infty$ ex. min ein Ableitungsbaum
\item für P ist $T_p^\infty$ das kleinste Herb. Modell\\$\rightarrow$ $F\in T_p^\infty \Leftrightarrow$ F in jeden Heb. Mod. vorkommt $\Leftrightarrow$ $P\models F$
\item Datenbankanfragen in präd. Log. ist PSpace-Complete\\(im Bezug auf Größe der Datenbank und Formel)
\item Zeitkomplexität: ExpTime
\item Bsp: Anfrage:\\$Q_o[x] = \exists x_0,z_{linie}.verbindung(x_0, x_1, z_{linie}) \land Q_0[x_0])$
\item gesuchte Var wird quantifiziert
\item bei Bool wird alles quantifiziert, das gesuchte mit $\exists$
\end{itemize}
\end{paracol}
\end{document}